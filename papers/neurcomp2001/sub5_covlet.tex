\documentclass [11pt]{letter}
\usepackage {times,epsf}
\oddsidemargin 0pt
\evensidemargin 0pt
\headheight 12pt
\headsep .5in
\topmargin -.75in
\footskip .75in
\textheight 9in
\textwidth 6.5in
\parskip 10pt
%\parindent 15pt

% pick one of the following
%\address{2585 Juniper Ave\\
%Boulder, CO  80304\\ (303) 448-1810\\}

%\address{Department of Psychology\\
%Campus Box 345\\
%University of Colorado at Boulder\\
%Boulder, CO 80309-0345\\ (303) 492-0054\\ (303) 492-2964 (fax)\\
%oreilly@psych.colorado.edu}

% for letterhead
\address{\vspace*{.5in}}

\signature{Randall O'Reilly\\Assistant Professor, Psychology\\
  oreilly@psych.colorado.edu}
%\signature{Randall O'Reilly}

\begin{document}

\begin{letter}
% replace address of sender here
{Dr. Terrence Sejnowski\\
The Salk Institute --- CNL\\
10010 North Torrey Pines Road\\
La Jolla, CA 92037\\}

\opening{Dear Dr. Sejnowski:}

Enclosed are 2 copies of manuscript number 1992 entitled,
``Generalization in Interactive Networks: The Benefits of Inhibitory
Competition and Hebbian Learning'' that has been accepted for
publication in {\em Neural Computation} as a {\em View}.  I have made
the minor revisions suggested by the first reviewer (Cottrell), and
have addressed the concerns of reviewer 3 by clarifying the use of the
``nonlinear'' and ``spurious'' attractor terminology, and how the
constraints in Leabra should produce better generalization.
Specifically, I added a new figure (fig 1) that demonstrates a very
simple example that makes the points more concrete (and maps directly
on to the subsequent simulations), and I related this work to the
distinction between conjunctive vs. elemental representations, which
more readers should be familiar with.  I then replaced the vaguer
terms of ``nonlinear'' and ``spurious'' with {\em conjunctive}, which
is more specifically the problem for combinatorial generalization.

I must add that I remain baffled by reviewer 2, who seems to not have
read my responses to his previous criticisms (or the paper itself), or
we have a very fundamental difficulty in communicating.  This reviewer
continues to miss the point that the paper is about interactive
networks being bad at generalization, and the fact that this is not
intended to be a universally beneficial set of biases, because we
already know that such things don't exist, and I have made ample
disclaimers to this effect in the paper, and reiterated them in the
cover letters.  In short, the reviewer is evaluating the paper
according to a set of standards which, aside from being theoretically
impossible, are not what is claimed in the paper.  I do not know where
these standards are coming from, but I have re-read the paper and
cannot find anything in it that is at all ambiguous on these points.
Therefore, it would appear to be pointless to continue iterating with
this reviewer.

\closing{Sincerely,}

\end{letter}
\end{document}

%%% Local Variables: 
%%% mode: latex
%%% TeX-master: t
%%% End: 
